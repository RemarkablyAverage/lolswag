% --------------------------------------------------------------
% This is all preamble stuff that you don't have to worry about.
% Head down to where it says "Start here"
% --------------------------------------------------------------
 
\documentclass[12pt]{article}
 
\usepackage[margin=1in]{geometry} 
\usepackage{amsmath,amsthm,amssymb}
 
\newcommand{\N}{\mathbb{N}}
\newcommand{\Z}{\mathbb{Z}}
 
\newenvironment{theorem}[2][Theorem]{\begin{trivlist}
\item[\hskip \labelsep {\bfseries #1}\hskip \labelsep {\bfseries #2.}]}{\end{trivlist}}
\newenvironment{lemma}[2][Lemma]{\begin{trivlist}
\item[\hskip \labelsep {\bfseries #1}\hskip \labelsep {\bfseries #2.}]}{\end{trivlist}}
\newenvironment{exercise}[2][Exercise]{\begin{trivlist}
\item[\hskip \labelsep {\bfseries #1}\hskip \labelsep {\bfseries #2.}]}{\end{trivlist}}
\newenvironment{problem}[2][Problem]{\begin{trivlist}
\item[\hskip \labelsep {\bfseries #1}\hskip \labelsep {\bfseries #2.}]}{\end{trivlist}}
\newenvironment{question}[2][Question]{\begin{trivlist}
\item[\hskip \labelsep {\bfseries #1}\hskip \labelsep {\bfseries #2.}]}{\end{trivlist}}
\newenvironment{corollary}[2][Corollary]{\begin{trivlist}
\item[\hskip \labelsep {\bfseries #1}\hskip \labelsep {\bfseries #2.}]}{\end{trivlist}}
 
\begin{document}
 
% --------------------------------------------------------------
%                         Start here
% --------------------------------------------------------------
 
\title{Total Swag Factor of Harold Pimentel}%replace X with the appropriate number
\author{Daniel Li\\ %replace with your name
Foundations of Harold's Swag} %if necessary, replace with your course title
 
\maketitle

\begin{introduction}
Inspiration taken from some upper division course I took on accident first semester past drop deadline without knowing what a PDE or partial derivative is.
\end{introduction}
\begin{theorem}{1} %You can use theorem, exercise, problem, or question here.  Modify x.yz to be whatever number you are proving
Harold's swag (Total Swag Factor) can be modeled with the following for any given instance:\\
$$\frac{\partial k}{\partial t} + \overline{u}_j \frac{\partial k}{\partial x_j} = - \frac{1}{\rho_0} \frac{\partial \overline{u'_ip'}}{\partial x_i} - \frac{1}{2} \frac{\partial \overline{u'_ju'_ju'_i}}{\partial x_i} + v \frac{\partial^2k}{\partial x^2_j} - \overline{u'_iu'j}\frac{\overline{u_i}}{\partial x_j} - v \overline{\frac{\partial u'_i}{\partial x_j}\frac{\partial u'_i}{\partial x_j}} - \frac{g}{\rho_0}\overline{p'u'_i}\delta_{i3} + \epsilon$$
\end{theorem}\\
\\

 
 \noindent\textbf{Definitions.}
Here, the previous equation's terms will be defined.
%Note 1: The * tells LaTeX not to number the lines.  If you remove the *, be sure to remove it below, too.
%Note 2: Inside the align environment, you do not want to use $-signs.  The reason for this is that this is already a math environment. This is why we have to include \text{} around any text inside the align environment.
\begin{align*}
k & = \text{Total Swag Factor (TSF)'s mean-flow} \\
\frac{\partial k}{\partial t} & = \text{Local material derivative of TSF with respect to time}\\
\overline{u}_j \frac{\partial k}{\partial x_j}
& = \text{Advection of TSF with influence onto others} \\
- \frac{1}{\rho_0} \frac{\partial \overline{u'_ip'}}{\partial x_i} & = 
\text{Diffusion of Knowledge}\\
- \frac{1}{2} \frac{\partial \overline{u'_ju'_ju'_i}}{\partial x_i} & = 
\text{Turbulent transport of swag}\\
 v \frac{\partial^2k}{\partial x^2_j} & = 
 \text{2nd moment of molecular swag}\\
- \overline{u'_iu'j}\frac{\overline{u_i}}{\partial x_j} & = \text{Swag absorbed due to gravity}\\
- v \overline{\frac{\partial u'_i}{\partial x_j}\frac{\partial u'_i}{\partial x_j}} & = 
\text{Swag loss due to biochemical processes of maintenance}\\
- \frac{g}{\rho_0}\overline{p'u'_i}\delta_{i3} & = 
\text{Swag flux}\\
\epsilon & = \text{Marginal error}
\end{align*}
 

% --------------------------------------------------------------
%     You don't have to mess with anything below this line.
% --------------------------------------------------------------
 
\end{document}
